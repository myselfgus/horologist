%%%%%%%%%%%%%%%%%%%%% capitulo2.tex %%%%%%%%%%%%%%%%%%%%%%%%%%%%%%%%%
%
% Capítulo 2 - A Cidade
%
%%%%%%%%%%%%%%%%%%%%%%%% Springer Nature %%%%%%%%%%%%%%%%%%%%%%%%%%

\chapter{A Cidade}
\label{cap:cidade}

O Relojoeiro, agora um peregrino do tempo, vagava por caminhos que só um louco ou um poeta ousaria trilhar, e quem, afinal, consegue distinguir entre os dois quando a razão já se desfez como açúcar em água quente?, ele caminhava por estradas que não constavam em mapa algum, que talvez nem mesmo existissem antes de seus pés tocarem o solo, criando-as no próprio ato de percorrê-las, assim como o tempo talvez só exista quando o nomeamos, quando o dissecamos em horas, minutos, segundos, essa taxonomia ridícula que pretende domesticar o indomesticável, classificar o inclassificável, dar nome ao que transcende a própria linguagem, e não seria isso, pensava ele com um sorriso de ironia que seus lábios finos mal conseguiam conter, a maior das presunções humanas? acreditar que podemos aprisionar em caixinhas mecânicas aquilo que nos contém, aquilo que nos engole, aquilo que, no fim, nos aniquila com a mesma indiferença com que o mar apaga pegadas na areia.

A carta misteriosa, que agora carregava no bolso interno do casaco como um amuleto ou uma maldição, havia conduzido seus passos até aqui, até esta encruzilhada onde o vento parecia soprar de todas as direções simultaneamente e onde as sombras se comportavam de maneira estranha, desobedecendo às leis mais elementares da física, como se o sol, esse grande e impiedoso cronometrista, tivesse perdido a autoridade sobre aquilo que iluminava, e foi então que viu a criança, parada no meio da estrada como uma aparição, uma anomalia temporal materializada em forma humana, havia algo de profundamente perturbador naquela figura pequena, não apenas os cabelos brancos como neve, que poderiam ser explicados por alguma condição médica rara (o Relojoeiro gostava de explicações, de causas e efeitos, de respostas que se encaixassem perfeitamente nas perguntas, assim como as engrenagens se encaixam umas nas outras, precisas, previsíveis, controláveis), mas sobretudo os olhos, aqueles olhos que pareciam ter testemunhado o nascimento e a morte de impérios, olhos que carregavam o peso de milênios comprimidos em íris de um azul tão saturado e profundo que mais pareciam fragmentos de céu aprisionados em carne mortal.

Bom dia, senhor que vem do lugar onde o tempo avança, disse a criança, e sua voz era como o ruído de folhas secas arrastadas pelo vento, como páginas antigas sendo viradas em uma biblioteca silenciosa, uma voz que não pertencia àquele corpo pequeno, uma voz que havia viajado através de eras para encontrá-lo naquela encruzilhada sem nome, no limite entre o conhecido e o incognoscível, entre o que pode ser medido e o que só pode ser experienciado, o Relojoeiro, em outras circunstâncias, teria ficado perturbado com essa saudação enigmática, com essa presunção de conhecimento sobre sua origem e destino, mas a verdade é que após dias vagando por paisagens cada vez mais estranhas, onde a solidez das coisas parecia dissolver-se gradualmente, onde as leis da natureza pareciam perder sua autoridade à medida que ele se afastava da cidade dos relógios, seu ceticismo havia começado a ceder lugar a uma curiosidade quase infantil, aquela mesma curiosidade que o havia levado, décadas atrás, a desmontar o primeiro relógio só para ver como funcionava por dentro, só para entender o que fazia os ponteiros se moverem com aquela precisão hipnótica, com aquela certeza inabalável de quem nunca questiona seu propósito, sua existência, sua razão de ser.

Você está procurando a Cidade das Horas Invertidas, não está?, perguntou a criança, não como uma questão real, mas como uma confirmação do que já sabia, e o Relojoeiro, surpreendendo a si mesmo, respondeu que sim, embora até aquele exato momento não soubesse que existia tal lugar, muito menos que o estava procurando, mas havia algo na maneira como a criança falou o nome daquele lugar, algo na cadência das sílabas, na ressonância das palavras, que despertou nele uma certeza inexplicável, como se um órgão interno desconhecido, um sentido adormecido, de repente tivesse despertado e agora reconhecesse a verdade quando a ouvia, porque há mentiras que soam falsas mesmo quando revestidas de todos os argumentos lógicos, assim como há verdades que soam incontestáveis mesmo quando desafiam toda a racionalidade, e a existência da Cidade das Horas Invertidas claramente pertencia a esta segunda categoria, uma verdade tão elementar que ele se perguntava como havia passado tantos anos sem cogitar sua possibilidade, seu nome, sua existência.

Então siga-me, disse a criança, estendendo uma mão pequena e enrugada, uma mão que parecia simultaneamente jovem e antiga, macia como a de um bebê e calejada como a de um trabalhador após décadas de labor, e o Relojoeiro, num gesto que contradisse toda sua cautela habitual, toda sua desconfiança metódica contra o que não podia explicar (aquela mesma desconfiança que o havia transformado em excelente relojoeiro, capaz de diagnosticar os problemas mais obscuros, de resolver os enigmas mecânicos mais complexos), tomou aquela mão em sua própria e permitiu-se ser guiado por um ser que desafiava tudo o que ele considerava possível, que zombava silenciosamente de suas concepções sobre o mundo, sobre o funcionamento das coisas, sobre a própria natureza da realidade, e talvez por isso mesmo fosse o guia perfeito para o que estava por vir, porque há momentos em que só podemos avançar quando abandonamos tudo o que pensávamos saber, quando nos entregamos à vertigem do desconhecido com a mesma confiança cega de quem se lança em um abismo sabendo que aprenderá a voar durante a queda, ou que descobrirá, no último instante, que o abismo nunca existiu.

À medida que se aproximavam da cidade, o Relojoeiro começou a notar pequenas anomalias na paisagem, sutis a princípio, depois cada vez mais evidentes, inquietantes, como se a própria estrutura da realidade começasse a se desmanchar nas bordas, a revelar sua natureza ilusória, árvores cujas sombras pareciam mover-se na direção errada, pássaros que voavam em trajetórias impossíveis, como se estivessem retrocedendo no tempo mas avançando no espaço, riachos que corriam da foz para a nascente, um lapso de lógica tão absurdo que ele quase riu, teria rido se não houvesse algo de profundamente perturbador naquelas inversões, algo que sugeria não um erro da natureza, mas uma natureza inteiramente diferente, operando sob princípios que seu cérebro, formatado por décadas de causalidade linear, de tempo unidirecional, mal conseguia processar sem sentir aquela vertigem particular que acompanha os momentos em que nossas certezas mais fundamentais começam a desmoronar, em que o edifício inteiro do que chamamos conhecimento revela-se como a construção frágil e arbitrária que sempre foi.

Bem-vindo à Cidade das Horas Invertidas, disse a criança quando chegaram a um portal de pedra coberto por símbolos que pareciam tanto runas antigas quanto equações matemáticas avançadas, e talvez fossem ambas, pensou o Relojoeiro, talvez no início e no fim de todas as coisas, a magia primitiva e a ciência mais refinada sejam apenas manifestações diferentes do mesmo impulso humano de compreender, de nomear, de controlar o incontrolável, talvez os antigos místicos e os físicos quânticos estejam falando do mesmo mistério fundamental usando vocabulários diferentes, rituais diferentes, mas perseguindo a mesma presa esquiva, que sempre escapa por entre os dedos no exato momento em que pensamos tê-la capturado, e há algo de profundamente cômico nessa busca eterna, nessa perseguição inútil, como um cachorro correndo atrás do próprio rabo, girando em círculos cada vez mais rápidos sem jamais perceber a futilidade de seu esforço, a impossibilidade de seu objetivo.

A cidade que se revelava diante dele parecia, à primeira vista, uma cidade comum, com casas, ruas, praças, pessoas circulando em seus afazeres diários, mas logo o Relojoeiro percebeu que algo estava fundamentalmente errado, ou talvez não errado, mas diferente, operando segundo uma lógica alternativa que fazia com que o familiar parecesse estranho, e o estranho, familiar, porque os idosos caminhavam com a energia e a agilidade de crianças, enquanto as crianças se moviam com a lentidão cautelosa e a sabedoria silenciosa dos anciãos, os edifícios mais novos pareciam desgastados, enquanto as ruínas antigas brilhavam com a solidez e o frescor de construções recentes, os jardins exibiam flores que murchavam ao serem tocadas pela luz do sol, como se a vida fosse um fardo do qual finalmente se libertavam, enquanto nas sombras, protegidas da claridade, novas flores brotavam com uma exuberância que parecia quase obscena, uma explosão de cores e formas que celebrava não o nascimento, mas a morte próxima, não o crescimento, mas o inevitável declínio que aguarda tudo que existe, que respira, que pulsa.

A senhora que o recebeu na primeira casa, para onde a criança o conduziu antes de desaparecer com a mesma inexplicabilidade com que havia surgido, era uma mulher de aparência jovem, quase adolescente, mas que se movia com a dificuldade de alguém que carrega o peso de muitas décadas, apoiando-se em uma bengala entalhada com símbolos semelhantes aos do portal, seus olhos, como os da criança, tinham aquela qualidade desconcertante de antiguidade, de conhecimento acumulado através de experiências que transcendiam uma única vida humana, ``Bom dia, jovem senhor,'' disse ela, com uma voz que oscilava entre um soprano e um contralto, como se não conseguisse decidir em qual registro estabelecer-se, ``ou será boa noite? Nunca sei ao certo, pois aqui, como deve ter notado, as coisas não seguem exatamente a ordem à qual você está acostumado, aqui o tempo nos devolve o que perdemos, aqui recordamos o futuro e antecipamos o passado, aqui as feridas cicatrizam antes de serem abertas e as lágrimas secam antes de serem derramadas'', e havia nisso, pensou o Relojoeiro, uma espécie de crueldade disfarçada de bênção, porque que tipo de vida é essa onde já se conhece o desfecho de cada história antes mesmo de seu início? que tipo de existência é essa onde a surpresa, o mistério, a descoberta são substituídos por uma certeza que esvazia de significado cada ação, cada escolha, cada momento?

``Sente-se,'' continuou a senhora, indicando uma cadeira que parecia nova e antiga ao mesmo tempo, como se estivesse simultaneamente no início e no fim de sua existência como objeto, ``tome um pouco de chá'', e ela serviu de uma chaleira fumegante um líquido que, ao cair na xícara, mudava de cor do escuro para o claro, do opaco para o transparente, ``e me conte o que o traz à nossa cidade, embora, é claro, eu já saiba a resposta, assim como sei o que dirá antes mesmo que você formule as palavras em sua mente'', e o Relojoeiro, irritado com essa presunção, com essa arrogância disfarçada de onisciência, decidiu mentir, dizer algo completamente inesperado, algo que nem ele mesmo havia pensado até aquele exato momento, mas quando abriu a boca, as palavras que saíram foram exatamente aquelas que a senhora já esperava, ``Vim em busca de respostas sobre o tempo'', e havia algo de humilhante em ser tão previsível, tão transparente, em não conseguir surpreender nem a si mesmo, porque talvez seja essa a maior armadilha do tempo linear ao qual estamos habituados, a ilusão de que somos livres, de que nossas escolhas são realmente nossas, quando na verdade talvez sejamos apenas atores seguindo um roteiro escrito muito antes de nascermos, marionetes dançando na ponta de fios invisíveis manipulados por mãos que não podemos ver, que nem mesmo sabemos se existem.

``O tempo não é o que parece'', disse a senhora, citando as exatas palavras da carta, e o Relojoeiro sentiu um arrepio subir por sua espinha, como se um dedo gelado tivesse tocado cada vértebra, uma a uma, num movimento ascendente que parecia querer conectar a terra ao céu usando seu corpo como condutor, ``e isso você já sabe, ou começou a suspeitar, ou por que outra razão teria deixado sua confortável oficina para vagar por aí como um lunático em busca de algo que nem mesmo sabe definir? Vocês, os que vivem no tempo linear, são tão ingênuos em sua arrogância, tão convictos de que compreendem o funcionamento do universo porque conseguem contar os segundos, medir os minutos, acumular as horas como quem acumula moedas numa caixa, sem jamais perceber que o tempo que vocês medem não é o mesmo tempo que vivem, assim como o mapa não é o território, a palavra não é a coisa, o nome não é o nomeado'', e havia, misturada àquela condescendência irritante, uma espécie de compaixão genuína, como se a senhora realmente se compadecesse de sua ignorância, de suas limitações perceptivas, da mesma forma que um adulto se compadece de uma criança que ainda acredita em monstros debaixo da cama (o que é irônico, pensou o Relojoeiro, porque a cidade inteira parecia uma coleção de monstros lógicos, de aberrações conceituais que desafiavam a própria natureza da realidade como ele a conhecia).

``Para entender a Cidade das Horas Invertidas,'' continuou a senhora, ``você deve primeiro desaprender tudo o que pensa saber sobre o tempo, deve abandonar a noção de que ele é uma flecha apontando sempre na mesma direção, do passado para o futuro, através desse ponto infinitesimal que chamamos de presente, deve considerar a possibilidade de que essa percepção unidirecional seja apenas uma limitação dos seus sentidos, das estruturas mentais impostas por uma evolução que privilegiou a sobrevivência imediata em detrimento da compreensão mais profunda, deve abraçar a ideia de que talvez o tempo seja como uma paisagem pela qual podemos caminhar em qualquer direção, onde todos os eventos --- o que você chama de passado, presente e futuro --- existem simultaneamente, apenas aguardando serem visitados'', e enquanto falava, seus olhos pareciam mudar de idade, alternando entre a vivacidade juvenil e o cansaço da velhice, como se estivesse saltando de uma extremidade a outra de sua própria linha do tempo, como se fosse muitas pessoas diferentes comprimidas em um único corpo, em uma única consciência, ou talvez, pensou o Relojoeiro com um sobressalto de compreensão, a mesma pessoa em muitos momentos diferentes, todos coexistindo no agora expandido, no presente que abarcava toda a eternidade.

``Venha,'' disse ela, levantando-se com a agilidade de quem acabou de descobrir que suas articulações podem, afinal, funcionar sem dor, ``vou mostrar a você o centro da nossa cidade, a Torre das Inversões, onde poderá ver por si mesmo o que estou tentando explicar'', e antes que pudesse protestar, dizer que acabara de chegar, que precisava descansar da longa jornada, que precisava tempo (e a ironia disso não lhe escapou) para processar tudo o que estava acontecendo, o Relojoeiro já se encontrava novamente nas ruas da cidade, guiado agora pela senhora que, a cada passo, parecia rejuvenescer ainda mais, seus cabelos escurecendo, suas rugas suavizando, sua postura endireitando, um processo de transformação contínua que deveria ser belo de se observar, mas que havia nele algo de profundamente perturbador, porque contrariava uma das poucas certezas que temos neste mundo de dúvidas: que o tempo só flui em uma direção, que envelhecemos, nunca rejuvenescemos, que caminhamos inexoravelmente para o fim, para o esquecimento, para o silêncio definitivo que aguarda todos nós, senhores e servos, sábios e tolos, relojoeiros e relógios.

A Torre das Inversões erguia-se no centro exato da cidade como um obelisco impossível, uma estrutura que parecia simultaneamente subir em direção ao céu e afundar-se na terra, um paradoxo arquitetônico que, como tudo naquele lugar, desafiava as noções mais básicas de realidade, em seu topo, embora ``topo'' fosse um conceito relativo naquela construção que parecia existir em mais dimensões do que as três habituais, havia um relógio monumental, diferente de qualquer outro que o Relojoeiro já tivesse visto ou consertado, porque seus ponteiros giravam no sentido anti-horário, marcando horas que pareciam avançar para trás, e mesmo assim, de alguma forma inexplicável, o tempo que indicava estava correto, não correto segundo as convenções arbitrárias das zonas horárias e dos meridianos, mas correto segundo uma verdade mais profunda, mais essencial, que ele podia sentir em seus ossos, em seu sangue, e foi nesse momento, olhando para aquele mecanismo impossível que funcionava perfeitamente, que o Relojoeiro sentiu seu mundo, sua compreensão, suas certezas desmoronarem como um castelo de cartas atingido por uma súbita corrente de ar.

``Ali está,'' disse sua guia, que agora parecia uma jovem de vinte anos, embora seus olhos ainda mantivessem aquela antiguidade, aquela profundidade que nenhum rejuvenescimento físico poderia apagar, ``o Grande Relógio Invertido, o coração mecânico de nossa cidade, a manifestação física do princípio que governa nossas vidas, nossas memórias, nossos destinos'', e o Relojoeiro não conseguia desviar o olhar daquele espetáculo hipnótico, daquela heresia mecânica que, no entanto, parecia mais honesta do que todos os relógios convencionais que já havia consertado, porque não pretendia aprisionar o tempo, mas libertá-lo de convenções arbitrárias, de limitações antropocêntricas, de simplificações redutoras que transformavam o infinitamente complexo em trivialmente simples, o misterioso em banal, o sagrado em profano.

``Mas que diabos é isso?'', murmurou ele para si mesmo, com a perplexidade típica de quem vê seu campo de especialidade, sua área de expertise, seu conhecimento acumulado ao longo de décadas ser desafiado, contradito, ridicularizado por algo que deveria ser impossível, mas que, ainda assim, funcionava diante de seus olhos com uma precisão que faria inveja aos mais finos relógios suíços, aos mais avançados cronômetros atômicos, ``como pode um relógio que vai para trás marcar a hora certa? como pode algo tão fundamentalmente errado funcionar tão impecavelmente?'', não havia lógica nisso, pensou, nenhuma explicação mecânica poderia justificar tal aberração, a menos que o problema não estivesse no relógio, mas na própria concepção de tempo, a menos que a inversão dos ponteiros fosse apenas a correção de um erro muito mais antigo, muito mais fundamental, um erro que remontava aos primeiros humanos que tentaram medir, classificar, domesticar a selvageria do tempo usando pedras e sombras, água e areia, rodas e engrenagens, cada novo método mais preciso que o anterior, mais refinado, mais elegante, e ainda assim, todos igualmente equivocados em sua premissa básica, todos igualmente limitados por uma percepção unidirecional que talvez seja apenas uma ilusão de ótica cósmica, uma falha perceptiva que nos impede de ver o tempo como ele realmente é: multidirecional, multidimensional, múltiplo em suas manifestações e singular em sua essência.

Desafiando seu próprio ceticismo, o Relojoeiro decidiu subir à torre, seguindo os passos de sua guia que, a essa altura, parecia quase uma criança, mas uma criança com a postura e a dignidade de uma anciã, uma combinação que seria cômica se não fosse tão inquietante, cada degrau que subia parecia levar tanto para cima quanto para dentro, como se estivesse simultaneamente escalando uma estrutura física e penetrando em camadas cada vez mais profundas de significado, de compreensão, de verdade, e quando finalmente chegou ao mecanismo principal, ao coração pulsante daquele organismo mecânico que desafiava tudo o que considerava sagrado em sua profissão, em sua arte, em sua ciência, encontrou um velho de olhar astuto sentado em um banco simples, contemplando o movimento dos ponteiros com a serenidade de quem observa um pôr do sol, um espetáculo natural que não requer intervenção, apenas apreciação silenciosa.

``Bem-vindo ao nosso tempo,'' disse o velho, sem erguer os olhos do relógio, sem oferecer qualquer explicação sobre como sabia da chegada do Relojoeiro, embora em um lugar onde a ordem convencional dos eventos parecia invertida, tal conhecimento prévio fosse mais regra do que exceção, ``tempo único, que corre ao contrário, que devolve juventude aos velhos e sabedoria às crianças, que nos permite lembrar do futuro e prever o passado, tempo circular, não linear, onde o fim é também o começo e cada momento contém em si todos os momentos'', o velho falava com a cadência de um poeta ou de um profeta, alguém acostumado a articular ideias que transcendem a linguagem comum, que exigem metáforas, analogias, parábolas para serem minimamente compreendidas, e mesmo assim permanecerão sempre parcialmente inacessíveis, como a própria experiência do tempo, que pode ser vivida mas nunca completamente descrita, sentida mas nunca totalmente comunicada.

``Como é possível viver assim,'' perguntou o Relojoeiro, mais por educação do que por genuína curiosidade, porque uma parte dele já havia começado a entender, a intuir as respostas antes mesmo de formular as perguntas, as peças do quebra-cabeça caindo em seus lugares com aquela inevitabilidade que caracteriza as grandes revelações, as epifanias que dividem a vida em ``antes'' e ``depois'', ``sem rumo, sem sentido, sabendo o que vai acontecer antes que aconteça, lembrando o que ainda não ocorreu, antecipando o que já passou?'', e havia em sua voz um misto de fascínio e horror, de atração e repulsa, a mesma ambivalência que sentimos diante do abismo, ao mesmo tempo temendo a queda e desejando secretamente o salto, a liberação, a entrega.

``Vivemos do mesmo modo que você,'' respondeu o velho com um sorriso que revelava a paciência infinita de quem já viu essa mesma cena repetir-se inúmeras vezes, de quem já ouviu essa mesma pergunta ser formulada por incontáveis viajantes desorientados que chegavam à Cidade das Horas Invertidas carregando suas concepções lineares, suas certezas frágeis, suas verdades provisórias, ``apenas invertemos a direção. Enquanto vocês esperam pelo desconhecido e lamentam o irreversível, nós aguardamos o inevitável e moldamos o que ainda não aconteceu. Enquanto vocês se surpreendem com cada novo momento e sofrem por não poder mudar o passado, nós saboreamos a certeza do que está por vir e celebramos a maleabilidade do que ainda não foi. Aqui, os erros se desfazem antes de serem cometidos, as dores se curam antes de serem sentidas. E as memórias? Ah, as memórias são construídas antes dos eventos, o que nos poupa da decepção'', e havia nessa explicação uma lógica tão impecável quanto perturbadora, uma coerência que o Relojoeiro queria refutar, mas não encontrava argumentos, porque como se pode argumentar contra uma experiência tão fundamentalmente diferente da sua própria? como se pode dizer que alguém está errado quando seus erros se desfazem antes de acontecerem? como se pode insistir na superioridade de uma temporalidade linear quando confrontado com a evidência viva, pulsante, inegável de que existem outras formas de experimentar, de viver, de ser no tempo?

Descendo da torre, mais perturbado do que esclarecido, mais confuso do que iluminado, o Relojoeiro encontrou novamente a menina de olhos antigos, aquela mesma que o havia guiado até a cidade, e havia algo de profundamente significativo nesse encontro, como se o círculo estivesse começando a se fechar, como se a narrativa estivesse lentamente retornando ao seu ponto de partida, não por falha ou repetição, mas por design, por necessidade estrutural, porque talvez as histórias, assim como o tempo naquela cidade, não sejam realmente lineares, mas circulares, não avancem realmente para um desfecho, mas retornem eternamente ao início, ganhando novas camadas de significado, novas possibilidades interpretativas a cada volta do ciclo.

``Para você,'' disse a menina, que talvez fosse a mesma senhora que o havia recebido, que talvez fosse o velho na torre, todos eles manifestações diferentes da mesma entidade ou consciência, fragmentos do mesmo ser refratado através do prisma do tempo invertido, estendendo para ele uma flor que, surpreendentemente, murchava diante de seus olhos, suas pétalas envelhecendo, escurecendo, encolhendo em questão de segundos, um processo que normalmente levaria dias comprimido em um instante, como se a proximidade do Relojoeiro, este representante do tempo linear, do tempo convencional, acelerasse seu declínio, catalisasse sua degeneração, ``aqui, as flores murcham antes de desabrochar, para nos lembrar que a beleza está na preservação do que ainda não aconteceu, na proteção do potencial contra sua inevitável realização'', e havia nesse gesto, nessa oferenda, nessa lição não solicitada, algo de profundamente comovente e simultaneamente irritante, porque quem eram eles, esses habitantes de uma cidade impossível, para questionar tudo o que ele sempre considerou verdadeiro, para desafiar as fundações mesmas de sua realidade, de sua existência, de sua compreensão do universo e de seu lugar nele?

Ainda assim, aceitou a flor murcha, guardando-a com cuidado no mesmo bolso onde carregava a carta enigmática, duas evidências físicas de que o mundo era mais estranho, mais complexo, mais fluido do que jamais havia imaginado, dois talismãs contra a tentação de retornar à confortável ignorância, à conveniência das certezas não examinadas, das verdades herdadas que aceitamos não porque as compreendemos, mas porque nos poupam do trabalho de questionar, de duvidar, de procurar por nós mesmos o que significa existir neste universo que parece, a cada nova descoberta, mais vasto e mais misterioso do que poderíamos conceber.

E assim, com a flor murcha em seu bolso e um turbilhão de novos pensamentos em sua mente, o Relojoeiro deixou a Cidade das Horas Invertidas, não porque tivesse obtido todas as respostas que buscava, mas porque intuía que a próxima etapa de sua jornada o aguardava em outro lugar, que o quebra-cabeça do tempo exigia mais peças, mais perspectivas, mais experiências antes que pudesse começar a formar uma imagem coerente, e havia nisso uma ironia que não lhe escapava: ele, que sempre viveu pela precisão, pela exatidão, pela certeza matemática dos mecanismos que consertava, agora se permitia ser guiado pela intuição, pelo instinto, por aquela bússola interna que não aponta para o norte magnético, mas para o norte espiritual, para o território desconhecido onde as respostas não são encontradas, mas criadas no próprio ato de buscar.

Como já havia aprendido, uma flor que murcha antes de desabrochar não é menos bela, apenas diferente; um relógio que marca o tempo ao contrário não é menos preciso, apenas obedece a outra lógica; uma vida que se lembra do futuro não é menos autêntica, apenas se orienta por coordenadas que nossos mapas convencionais não registram, e havia nessa aceitação da diferença, nessa abertura ao impensável, uma liberdade que nunca havia experimentado antes, uma expansão de consciência que tornava sua antiga existência, tão estruturada, tão previsível, tão regulada pelo tic-tac monótono dos relógios convencionais, simultaneamente confortável em sua simplicidade e insuportavelmente restritiva em suas limitações.

\begin{center}
$\ast$~$\ast$~$\ast$
\end{center}

E assim, caro leitor, voltamos ao início, ao ponto onde o Relojoeiro, agora com uma nova flor murcha nas mãos e um turbilhão de pensamentos na mente, reabre as cortinas de sua oficina. A luz do sol entra, como sempre, mas algo dentro dele havia mudado. A inquietação agora tinha um nome, um propósito, um destino a seguir. Como no início de sua jornada, ele caminha pelas ruas adormecidas da cidade, mas desta vez com a certeza de que o tempo é mais do que um conjunto de engrenagens, é uma tapeçaria de perguntas que, quando respondidas, apenas geram novas perguntas, mais profundas, mais essenciais, mais próximas daquela verdade fundamental que sempre nos escapa, que sempre nos convida a continuar buscando, questionando, duvidando.

Era uma manhã como outra qualquer, ou assim parecia ao Relojoeiro, que, ao abrir as cortinas de sua oficina, deixou que os primeiros raios de sol invadissem o espaço repleto de relógios e sombras. Mas algo dentro dele havia mudado, uma inquietação, um formigamento de pensamentos que não se aquietavam, que não se contentavam com o tic-tac monótono dos ponteiros. E foi então que ele decidiu, com a mesma certeza com que se sabe que o dia segue a noite, que era hora de buscar respostas, era hora de entender o tempo, não como um conjunto de engrenagens e mostradores, mas como a essência que permeia tudo, que define a vida e a morte, o amor e a perda, o encontro e a despedida.

E assim, caro leitor, você percebe que estamos presos em um ciclo, um eterno retorno, uma brincadeira cruel do tempo. A cada descoberta, voltamos ao ponto de partida, e a cada volta, a compreensão se aprofunda. Porque o tempo, essa entidade caprichosa, adora pregar peças, adora nos fazer pensar que estamos avançando quando, na verdade, estamos apenas circulando ao redor da verdade. E é nesse movimento perpétuo que reside a beleza da busca, a essência do tempo.

\begin{center}
\textit{A flor murcha da Cidade das Horas Invertidas}
\end{center}
